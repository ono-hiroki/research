\expandafter\ifx\csname ifdraft\endcsname\relax
\documentclass{jsarticle}
\begin{document}
    \fi
    \section{ライブラリの使用と改善点の発見}
    この章では、ライブラリを使用したことによって発見した改善点について述べる。
    \subsection{ライブラリの使用}
    何を作ったか、どういうライブラリを使ったか、どういうコードを書いたかを書く。
    言うても、いらないかもしれない。

    \subsection{改善点の発見}
    本研究では、ライブラリを使用したことによって、以下の改善点を発見した。

\begin{lstlisting}[caption=hoge,label=fuga]
// このとき、FloatとIntegerのレイヤの活性化は排反でよい気がする
const [getHoge, setHoge] = useLayerParams('', ["Float", "Integer"]);

// いちいち切り替えがめんどう
layerManager.deactivateLayer("Integer");
layerManager.activateLayer("Float");


// 排反ではないから2つ条件含むのどうなん?
<Layer condition={layerState.Float && !layerState.Integer}>
\end{lstlisting}



    \expandafter\ifx\csname ifdraft\endcsname\relax
\end{document}
\fi
