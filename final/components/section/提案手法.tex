\expandafter\ifx\csname ifdraft\endcsname\relax
\documentclass{jsarticle}
\begin{document}
\fi
\section{提案手法}
本研究では、関連研究にある複合層、多層の機能追加
react copの改善点を洗い出し、それを解決するための機能を追加した。
本章では、本研究で提案する手法について述べる。

react copでは、レイヤーの活性化情報をuseStateを用いて管理している。state変数はクラスのインスタンスを用いている。
しかし、state変数は参照型のため、state変数の値を変更しても再レンダリングが行われない。
そのため、レイヤーの活性化情報を更新しても、再レンダリングが行われない。
これを解決するために、state変数を参照型のクラスのインスタンスから、イミュータブルなデータに変更する必要がある。
この変更はreact copを大きく刷新することと同等であるため、react cop2として新しく実装することとした。

react copでは、レイヤーのパラメーターとlayerの活性化情報を別で管理している。
そのため、レイヤーの活性化とレイヤーのパラメーターで2度レイヤー名を指定する必要がある。
これは、レイヤーの管理が煩雑になる原因となっている。
この問題を解決するために、レイヤーを一つのオブジェクトとして管理することとした。

以降では、react cop2の実装について述べる。
ReactCOPで代替となる機能を提供するために、react cop2では、以下の機能を提供する。
ReactCOP2では、カスタムコンテキストを用いて、レイヤーの操作や取得などの機能を提供する。
react copから改善された機能を以下に示す。

%     layerの管理の方法を大きく変更した。
%     同じくuseStateを用いてlayerの管理を行うが、state変数を参照型のクラスのインスタンスから、イミュータブルなデータ構造である配列に変更した。
%     state変数を参照型にしてしまうと、state変数の値を変更が検知できないため、再レンダリングが行われない。
%     しかし、イミュータブルなデータ構造である配列に変更することで、state変数の値の変更を検知できるようになった。


\subsection{代替する機能}
\subsubsection{レイヤーパラムの設定・取得}
React copでは、レイヤーパラムの設定・取得ができるuseLayerParamsというカスタムフックを提供している。
このカスタムフックは、以下のように使用する。
\begin{lstlisting}[]
// レイヤーパラム
const [getHoge, setHoge] = useLayerParams('hoge', ["Hoge"]);
getHoge() // hoge
setHoge('fuga',"Hoge")
getHoge() // fuga

\end{lstlisting}
useLayerParamsの第一引数には、レイヤーパラムの初期値、第二引数には、レイヤー名を指定することで、レイヤーパラムの値を設定できる。
レイヤー名は複数指定することができる。
setHogeの第一引数には、レイヤーパラムの値、第二引数には、レイヤー名を指定することでレイヤーパラムの値を設定できる。
getHogeの引数には、レイヤー名を指定することで、レイヤーパラムの値を取得できる。
引数にレイヤー名を指定しない場合は、活性化したレイヤーのレイヤーパラムの値を取得する。
ただし、活性化したレイヤーが複数ある場合は、最初に取得したレイヤーのレイヤーパラムの値を取得する。
レイヤーの並び順は、レイヤー名を登録した順番である。

ReactCOP2では、レイヤーパラムの設定ができるsetLayerParams、取得ができるgetLayerParamsというメソッドを提供する。
このメソッドは、以下のように使用する。
% TODO: ここにコードを書く
setLayerParamsの第一引数には、レイヤー名、第二引数にはレイヤーパラムの値を指定することで、レイヤーパラムの値を設定できる。
getLayerParamsの引数には、レイヤー名を指定することで、レイヤーパラムの値を取得できる。

ReactCOP2では、レイヤーパラムの設定ができる
\subsubsection{レイヤーの活性化・非活性化}
ReactCOPでは、レイヤーの活性化・非活性化ができるuseLayerManagerというカスタムフックを提供している。
このカスタムフックは、以下のように使用する。
\begin{lstlisting}[]
const layerManager = useLayerManager();
layerManager.activateLayer("Float");
layerManager.deactivateLayer("Integer");
\end{lstlisting}
activateLayerの引数には、レイヤー名を指定することで、レイヤーを活性化できる。
deactivateLayerの引数には、レイヤー名を指定することで、レイヤーを非活性化できる。

\subsubsection{レイヤーの活性化情報の取得}
ReactCOPでは、レイヤーの活性化情報を取得できるuseLayerManagerというカスタムフックはgetLayerStateというメソッドを提供している。
このメソッドは、以下のように使用する。
\begin{lstlisting}[]
const layerManager = useLayerManager();
const layerState = layerManager.getLayerState();

layerState.Float
layerState.Integer
\end{lstlisting}
layerState.{レイヤー名}で、レイヤーの活性化情報を取得できる。
ただし、あらかじめレイヤーを活性化・非活性化していないと、レイヤーの活性化情報は取得でず、undefinedが返される。
\subsubsection{レイヤーが活性化しているかどうかの判定}
ReactCOPでは、レイヤーが活性化しているかどうかの判定ができるuseLayerManagerというカスタムフックはisActiveLayerというメソッドを提供している。
このメソッドは、以下のように使用する。
\begin{lstlisting}[]
const layerManager = useLayerManager();
// レイヤーがactiveかどうかを判定
layerManager.isActiveLayer("Float")// true or false
\end{lstlisting}
isActiveLayerの引数には、レイヤー名を指定することで、レイヤーが活性化しているかどうかを判定できる。
レイヤーが活性化している場合は、trueが返される。
\subsection{改善点}

\subsubsection{改善点一覧}

\begin{itemize}
	\item layerのde/active時に新しいレイヤーを定義できないようにする
	      react copでは、layerのde/active時に新しいレイヤーを定義できてしまう。
	      そのため、意図しないレイヤーが簡単に定義できてしまう。
	      またレイヤーの管理が煩雑になる。
	      コレを解決するために、layerのde/active時に新しいレイヤーを定義できないようにする
	\item layer paramsはin/activeの両方の状態を持つ
	\item layer paramsはlayerのin/active状態に依存をするようにしたい
	\item layer paramsの値を入れるときに新しいlayerを定義できないようにする
	\item layer grop的なのをついか
	\item layerの活性化条件を定義できる
	      \begin{itemize}
		      \item 複合層
		      \item 多層
	      \end{itemize}
	\item layerの活性化は排反
	\item typescriptでの実装
	\item テストの追加
\end{itemize}

\subsubsection{layerのde/active時に新しいレイヤーを定義できないようにする}
ReactCOPでは、layerのde/active時に新しいレイヤーを定義できてしまう。
layerのde/active時に新しいレイヤーを定義できると意図しないレイヤーが簡単に定義できてしまう。
またレイヤーの管理が煩雑になる。

ReactCOP2では、layerのde/active時に新しいレイヤーを定義できないようにする。
これによって、意図しないレイヤーが簡単に定義できなくなり、レイヤーの管理が煩雑にならない。


\subsubsection{layer paramsはin/activeの両方の状態を持つ}

\subsubsection{layer paramsはlayerのin/active状態に依存をするようにしたい}

\subsubsection{layer paramsの値を入れるときに新しいlayerを定義できないようにする}

\subsubsection{layer grop的なのをついか}

\subsubsection{layerの活性化条件を定義できる}

\subsubsection{layerの活性化は排反}
\begin{lstlisting}[caption=hoge,label=fuga]
// このとき、FloatとIntegerのレイヤの活性化は排反でよい気がする
const [getHoge, setHoge] = useLayerParams('', ["Float", "Integer"]);

// いちいち切り替えがめんどう
layerManager.deactivateLayer("Integer");
layerManager.activateLayer("Float");


// 排反ではないから2つ条件含むのどうなん?
<Layer condition={layerState.Float && !layerState.Integer}>
\end{lstlisting}

\subsection{実装する内容}

\subsubsection{typescriptでの実装}

\subsubsection{テストの追加}


\subsection{評価方法}
- 実装前と後で、できることの違いを比較する。

\expandafter\ifx\csname ifdraft\endcsname\relax
\end{document}
\fi
