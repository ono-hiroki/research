\expandafter\ifx\csname ifdraft\endcsname\relax
\documentclass{jsarticle}
\begin{document}
\fi
\section{提案手法}
本研究では、関連研究にある複合層、多層の機能追加
react copの改善点を洗い出し、それを解決するための機能を追加した。
本章では、本研究で提案する手法について述べる。

ReactCOPでは、レイヤーの活性化情報をuseStateを用いて管理している。state変数はクラスのインスタンスを用いている。
しかし、state変数は参照型のため、state変数の値を変更しても再レンダリングが行われない。
そのため、レイヤーの活性化情報を更新しても、再レンダリングが行われない。
これを解決するために、state変数を参照型のクラスのインスタンスから、イミュータブルなデータに変更する必要がある。
この変更はreact copを大きく刷新することと同等であるため、ReactCOP2ではとして新しく実装することとした。

ReactCOPでは、レイヤーのパラメーターとlayerの活性化情報を別で管理している。
そのため、レイヤーの活性化とレイヤーのパラメーターで2度レイヤー名を指定する必要がある。
これは、レイヤーの管理が煩雑になる原因となっている。
この問題を解決するために、レイヤーを一つのオブジェクトとして管理することとした。

以降では、ReactCOP2の実装について述べる。
ReactCOPで代替となる機能を提供するために、react cop2では、以下の機能を提供する。
ReactCOP2では、カスタムコンテキストを用いて、レイヤーの操作や取得などの機能を提供する。
react copから改善された機能を以下に示す。

%     layerの管理の方法を大きく変更した。
%     同じくuseStateを用いてlayerの管理を行うが、state変数を参照型のクラスのインスタンスから、イミュータブルなデータ構造である配列に変更した。
%     state変数を参照型にしてしまうと、state変数の値を変更が検知できないため、再レンダリングが行われない。
%     しかし、イミュータブルなデータ構造である配列に変更することで、state変数の値の変更を検知できるようになった。


\subsection{代替する機能}
\subsubsection{レイヤーパラムの設定・取得}
React copでは、レイヤーパラムの設定・取得ができるuseLayerParamsというカスタムフックを提供している。
このカスタムフックは、以下のように使用する。
\begin{lstlisting}[]      
// レイヤーパラム
const [getHoge, setHoge] = useLayerParams('hoge', ["Hoge"]);
getHoge() // hoge
setHoge('fuga',"Hoge")
getHoge() // fuga

\end{lstlisting}
useLayerParamsの第一引数には、レイヤーパラムの初期値、第二引数には、レイヤー名を指定することで、レイヤーパラムの値を設定できる。
レイヤー名は複数指定することができる。もしレイヤーが存在しない場合は、新しくレイヤーを定義する。これは、意図しないレイヤーが簡単に定義できてしまう。
setHogeの第一引数には、レイヤーパラムの値、第二引数には、レイヤー名を指定することでレイヤーパラムの値を設定できる。
getHogeの引数には、レイヤー名を指定することで、レイヤーパラムの値を取得できる。
引数にレイヤー名を指定しない場合は、活性化したレイヤーのレイヤーパラムの値を取得する。
ただし、活性化したレイヤーが複数ある場合は、最初に取得したレイヤーのレイヤーパラムの値を取得する。
レイヤーの並び順は、レイヤー名を登録した順番である。

ReactCOP2では、レイヤーパラムの設定ができるsetLayerParams、取得ができるgetLayersParamというメソッドを提供する。
このメソッドは、以下のように使用する。
% TODO: ここにコードを書く
setLayerParamsの第一引数には、レイヤー名、第二引数にはレイヤーパラムの値を指定することで、レイヤーパラムの値を設定できる。
getLayerParamsの引数には、レイヤー名を指定することで、レイヤーパラムの値を取得できる。

\subsubsection{レイヤーの活性化・非活性化}
ReactCOPでは、レイヤーの活性化・非活性化ができるuseLayerManagerというカスタムフックを提供している。
このカスタムフックは、以下のように使用する。
\begin{lstlisting}[]
const layerManager = useLayerManager();
layerManager.activateLayer("Float");
layerManager.deactivateLayer("Integer");
\end{lstlisting}
activateLayerの引数には、レイヤー名を指定することで、レイヤーを活性化できる。
deactivateLayerの引数には、レイヤー名を指定することで、レイヤーを非活性化できる。

ReactCOP2では、レイヤーの活性化・非活性化ができるactivateLayer、inactivateLayerというメソッドを提供する。
このメソッドは、以下のように使用する。
% TODO: ここにコードを書く
activateLayerの引数には、レイヤー名を指定することで、レイヤーを活性化できる。
inactivateLayerの引数には、レイヤー名を指定することで、レイヤーを非活性化できる。

\subsubsection{レイヤーの活性化情報の取得}
ReactCOPでは、レイヤーの活性化情報を取得できるuseLayerManagerというカスタムフックはgetLayerStateというメソッドを提供している。
このメソッドは、以下のように使用する。
\begin{lstlisting}[]
const layerManager = useLayerManager();
const layerState = layerManager.getLayerState();

layerState.Float
layerState.Integer
\end{lstlisting}
layerState.{レイヤー名}で、レイヤーの活性化情報を取得できる。
ただし、あらかじめレイヤーを活性化・非活性化していないと、レイヤーの活性化情報は取得でず、undefinedが返される。

ReactCOP2では、レイヤーの活性化情報を取得できるgetLayerというメソッドを提供する。% TODO: getLayerメソッドを追加する
このメソッドは、以下のように使用する。
% TODO: ここにコードを書く
getLayerの引数には、レイヤー名を指定することで、レイヤーの情報を取得できる。

\subsubsection{レイヤーが活性化しているかどうかの判定}
ReactCOPでは、レイヤーが活性化しているかどうかの判定ができるuseLayerManagerというカスタムフックはisActiveLayerというメソッドを提供している。
このメソッドは、以下のように使用する。
\begin{lstlisting}[]
const layerManager = useLayerManager();
// レイヤーがactiveかどうかを判定
layerManager.isActiveLayer("Float")// true or false
\end{lstlisting}
isActiveLayerの引数には、レイヤー名を指定することで、レイヤーが活性化しているかどうかを判定できる。
レイヤーが活性化している場合は、trueが返される。

ReactCOP2では、レイヤーが活性化しているかどうかの判定ができるメソッドは提供しない。
代わりに、レイヤーの活性化情報を取得できるgetLayerメソッドから、レイヤーが活性化情報を取得すれば、レイヤーが活性化しているかどうかの判定ができる。
TODO: ここにコードを書く

\subsection{改善点}


\subsubsection{改善点一覧}

\begin{itemize}
	\item layerのde/active時に新しいレイヤーを定義できないようにする
	      react copでは、layerのde/active時に新しいレイヤーを定義できてしまう。
	      そのため、意図しないレイヤーが簡単に定義できてしまう。
	      またレイヤーの管理が煩雑になる。
	      コレを解決するために、layerのde/active時に新しいレイヤーを定義できないようにする
	\item layer paramsの値を入れるときに新しいlayerを定義できないようにする
	\item typescriptでの実装
	\item テストの追加
\end{itemize}


\subsubsection{layerのde/active時に新しいレイヤーを定義できないようにする}
ReactCOPでは、layerのde/active時に新しいレイヤーを定義できてしまう。
layerのde/active時に新しいレイヤーを定義できると意図しないレイヤーが簡単に定義できてしまう。
またレイヤーの管理が煩雑になる。

ReactCOP2では、layerのde/active時に新しいレイヤーを定義できないようにする。
これによって、意図しないレイヤーが簡単に定義できなくなり、レイヤーの管理が煩雑にならない。


\subsubsection{layerの活性化条件を定義できる}

% \subsubsection{layerの活性化は排反}
% \begin{lstlisting}[caption=hoge,label=fuga]
% // このとき、FloatとIntegerのレイヤの活性化は排反でよい気がする
% const [getHoge, setHoge] = useLayerParams('', ["Float", "Integer"]);

% // いちいち切り替えがめんどう
% layerManager.deactivateLayer("Integer");
% layerManager.activateLayer("Float");


% // 排反ではないから2つ条件含むのどうなん?
% <Layer condition={layerState.Float && !layerState.Integer}>
% \end{lstlisting}



\subsection{機能追加}

\subsubsection{機能追加一覧}

\begin{itemize}
	\item アクティブなレイヤーを取得する
		getActiveLayers: () => Layers;
	\item レイヤーのパラメーターを取得する
		getLayersParam: (names: string[]) => any[];
		getAllLayersParams: () => any;
		getActiveLayersParams: () => any;
		getLayerParamsByGroup: (group: string) => any;
	\item グループを指定してレイヤーを取得する
		getLayersByGroup: (group: string) => Layers;
		getActiveLayersByGroup: (group: string) => Layers;
		getInactiveLayersByGroup: (group: string) => Layers;

% evaluateDependencies: (dependencies: DependencyGroup) => boolean;
% applyDependenciesToAllLayers: (layers: Layers) => Layers;

% セットする系
	\item レイヤーを追加する
		addLayer: (layer: Layer) => void;
	\item レイヤーを削除する
		removeLayer: (name: string) => void;
	\item レイヤーの活性化を切り替える
		toggleLayer: (name: string) => void;
	\item グループの設定をする
		setGroupConfig: (name: string, config: GroupConfig) => void;
	\item レイヤーにグループを設定する
		setLayerGroup: (name: string | string[], group: string) => void;
	\item レイヤーの活性化条件を設定する
		% setLayerCondition: (name: string, condition: (layers: Layers) => boolean) => void;
		addDependency: (name: string, dependencies: DependencyGroup) => void;		
\end{itemize}

\subsection{アクティブなレイヤーを取得する}
ReactCOPでアクティブなレイヤーを取得するためには、レイヤーの活性化情報を取得し、活性化しているレイヤーを取得する必要がある。
これは、レイヤーの管理が煩雑になる原因となっている。
この問題を解決するために、アクティブなレイヤーを取得する機能を提供することで、レイヤーの管理が簡単になる。

% TODO: コードでの比較をする

\subsection{レイヤーのパラメーターを取得する}
\subsubsection{複数のレイヤーのパラメーターを取得する}
ReactCOPでは、複数のレイヤーのパラメーターを取得するためには、レイヤーのパラメーターを取得するためには、レイヤー名を指定してレイヤーのパラメーターを取得する必要がある。
少し不便であるため、複数のレイヤーのパラメーターを取得する機能を提供することで、レイヤーの管理が簡単になる。

\subsubsection{全てのレイヤーのパラメーターを取得する}
ReactCOPでは、全てのレイヤーのパラメーターを取得するためには、レイヤーのパラメーターを取得するためには、レイヤー名を指定してレイヤーのパラメーターを取得する必要がある。
少し不便であるため、全てのレイヤーのパラメーターを取得する機能を提供することで、レイヤーの管理が簡単になる。

\subsubsection{アクティブなレイヤーのパラメーターを取得する}
ReactCOPでアクティブなレイヤーのパラメーターを取得するためには、レイヤーが活性化しているかどうかの判定を行い、活性化しているレイヤーのパラメーターを取得する必要がある。
これは、少し不便であるため、アクティブなレイヤーのパラメーターを取得する機能を提供することで、レイヤーの管理が簡単になる。

\subsubsection{グループを指定してレイヤーのパラメーターを取得する}
RectCOPでは、そもそもグループの概念がないため、グループを指定してレイヤーのパラメーターを取得する機能は提供していない。
しかし、グループを指定してレイヤーのパラメーターを取得する機能を提供することで、レイヤーの管理が簡単になる。

\subsection{レイヤーを追加する}
ReactCOPでは、明確にレイヤーを追加する機能を提供していない。
レイヤーを活性化させるときやレイヤーのパラメーターを設定するときに、もしレイヤーが存在しない場合はレイヤーを追加するという処理を行っている。
これは、いつレイヤーが追加されるかわからないため、レイヤーの管理が煩雑になる。

ReactCOP2では、明確にレイヤーを追加する機能を提供することで、レイヤーの管理が簡単になる。

\subsection{レイヤーを削除する}
必要がないレイヤーを削除することで、レイヤーの管理が簡単になる。

\subsection{レイヤーの活性化を切り替える}
ReactCOPでは、レイヤーの活性化を切り替える機能を提供していない。
レイヤーの活性化を切り替えるには、レイヤーの活性化情報を取得し、活性化しているかどうかの判定を行い、活性化している場合は、レイヤーを非活性化し、非活性化している場合は、レイヤーを活性化する必要がある。
これは、少し手間であるため、レイヤーの活性化を切り替える機能を提供することで、レイヤーの管理が簡単になる。

\subsection{グループの設定をする}
ReactCOPでは、そもそもグループの概念がないため、グループの設定をする機能は提供していない。
グループとは、レイヤーをグループ化することである。
グループを設定することで、グループ内のレイヤー内で活性化するレイヤーを取得するなどの機能を提供することができる。

ReactCOP2では、グループの設定をする機能を提供することで、レイヤーの管理が簡単になる。

\subsection{レイヤーにグループを設定する}
レイヤーがどのグループに属するか設定することができる。

\subsection{レイヤーの活性化条件を設定する}
レイヤーの活性化条件を設定することができる。
レイヤーの活性化条件とは、レイヤーが活性化する条件である。
例えば、レイヤーAとレイヤーBがあるとき、レイヤーAが活性化しているときに、レイヤーBを活性化するという条件を設定することができる。
条件の指定方法は、レイヤーの活性化状態の論理積、論理和を指定できることである。

\subsubsection{Typescriptでの実装}
ReactCOPでは、Typescriptでの実装をしていない。
Typescriptでの実装をすることで、コードの可読性と保守性が向上する。

\subsubsection{テストの追加}
ReactCOPでは、テストをしていない。
テストは、コードの品質を保つために必要である。

\subsection{評価方法}
- 実装前と後で、できることの違いを比較する。

\expandafter\ifx\csname ifdraft\endcsname\relax
\end{document}
\fi
