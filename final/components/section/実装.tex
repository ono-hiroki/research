\expandafter\ifx\csname ifdraft\endcsname\relax
\documentclass{jsarticle}
\begin{document}
\fi
\section{実装}
大まかな構成はReactCOPと同じである。
useStateを用いてレイヤー情報を管理する。
レイヤー情報はProviderを用いて、子コンポーネントに渡す。
子コンポーネントは、useContextを用いてレイヤー情報を取得する。

\subsection{レイヤーの型}
レイヤーの型は、以下のように定義する。
\begin{lstlisting}[caption=レイヤーの型]
export interface Layer {
    name: string;
    isActive: boolean; // state
    group: string;
    params: any;
    dependencies?: DependencyGroup;
}
\end{lstlisting}
nameはレイヤーの名前である。
isActiveはレイヤーが活性化しているかどうかを表す。
groupはレイヤーのグループ名である。
paramsはレイヤーパラムである。
dependenciesはレイヤーが活性化する条件である。

\subsection{代替する機能}
\subsubsection{レイヤーパラムの設定・取得}
\subsubsection{ひとつのレイヤーのレイヤーパラムを取得}
ReactCOPでは、レイヤーパラムの設定・取得ができるgetLayerParamというメソッドを提供している。
型は以下のように定義する。
\begin{lstlisting}[caption=getLayerParamの型]
getLayerParam: (name: string) => any;
\end{lstlisting}
レイヤー名を指定することで、レイヤーパラムの値を取得できる。
レイヤーパラムは何を設定しても良いため、any型である。

このメソッドは、以下のように使用する。
\begin{lstlisting}[]
getLayerParam('hoge')
\end{lstlisting}

\subsubsection{複数のレイヤーのレイヤーパラムを取得}
複数のレイヤーパラムを取得したい場合はgetLayersParamというメソッドを提供している。
% TODO ここに型を書く つぎはここから書く

このメソッドは、以下のように使用する。
\begin{lstlisting}[]
getLayersParam(['hoge','fuga'])
\end{lstlisting}


\expandafter\ifx\csname ifdraft\endcsname\relax
\end{document}
\fi
