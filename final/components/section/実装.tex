\expandafter\ifx\csname ifdraft\endcsname\relax
\documentclass{jsarticle}
\begin{document}
\fi
\section{実装}
大まかな構成はReactCOPと同じである。
useStateを用いてレイヤー情報を管理する。
レイヤー情報はProviderを用いて、子コンポーネントに渡す。
子コンポーネントは、useContextを用いてレイヤー情報を取得する。

\subsection{レイヤーの型}
レイヤーの型は、以下のように定義する。
\begin{lstlisting}[caption=レイヤーの型]
export interface Layer {
    name: string;
    isActive: boolean; // state
    group: string;
    params: any;
    dependencies?: DependencyGroup;
}
\end{lstlisting}
nameはレイヤーの名前である。
isActiveはレイヤーが活性化しているかどうかを表す。
groupはレイヤーのグループ名である。
paramsはレイヤーパラムである。
dependenciesはレイヤーが活性化する条件である。

\subsection{代替する機能}
\subsubsection{レイヤーパラムの設定・取得}

\subsubsection{レイヤーパラムの設定}
レイヤーパラムの設定はsetLayerParamsというメソッドを提供している。
型は以下のように定義する。
\begin{lstlisting}[caption=setLayerParamsの型]
setLayerParams: (name: string, params: Record<string, any>) => void;
\end{lstlisting}
レイヤー名とレイヤーパラムを指定することで、レイヤーパラムの値を設定できる。
レイヤーパラムはレイヤーのparamsに格納される。
ReactCOPではレイヤーが存在しない場合は、新しくレイヤーを定義でき、意図しないレイヤーが混入する可能性があった。
ReactCOP2では、レイヤーが存在しない場合は、エラーを返すようにした。

\subsubsection{ひとつのレイヤーのレイヤーパラムを取得}
ReactCOPでは、レイヤーパラムの設定・取得ができるgetLayerParamというメソッドを提供している。
型は以下のように定義する。
\begin{lstlisting}[caption=getLayerParamの型]
getLayerParam: (name: string) => any;
\end{lstlisting}
レイヤー名を指定することで、レイヤーパラムの値を取得できる。
レイヤーパラムは何を設定しても良いため、any型である。

このメソッドは、以下のように使用する。
\begin{lstlisting}[]
getLayerParam('hoge')
\end{lstlisting}

\subsubsection{複数のレイヤーのレイヤーパラムを取得}
複数のレイヤーパラムを取得したい場合はgetLayersParamというメソッドを提供している。
% TODO ここに型を書く つぎはここから書く
型は以下のように定義する。
\begin{lstlisting}[caption=getLayersParamの型]
getLayersParam: (names: string[]) => any[];
\end{lstlisting}
レイヤー名を配列で指定することで、レイヤーパラムの値を配列で取得できる。
レイヤーパラムは何を設定しても良いため、any型である。

このメソッドは、以下のように使用する。
\begin{lstlisting}[]
getLayersParam(['hoge','fuga'])
\end{lstlisting}

\subsubsection{レイヤーの活性化・非活性化}
レイヤーの活性化・非活性化は、activateLayerとinactivateLayerというメソッドを提供している。
型は以下のように定義する。
\begin{lstlisting}[caption=activateLayerとinactivateLayerの型]
activeLayer: (name: string) => void; 
inactiveLayer: (name: string) => void;
\end{lstlisting}
レイヤー名を指定することで、レイヤーの活性化・非活性化ができる。
具体的にはレイヤーのisActiveを活性化状態ならtrue、非活性化状態ならfalseにする。
ReactCOPではレイヤーが存在しない場合は、新しくレイヤーを定義でき、意図しないレイヤーが混入する可能性があった。
ReactCOP2では、レイヤーが存在しない場合は、エラーを返すようにした。

\subsubsection{レイヤーの活性化情報の取得}
レイヤーの活性化情報を取得したい場合はgetLayerというメソッドを提供している。
型は以下のように定義する。
\begin{lstlisting}[caption=getLayerの型]
getLayer: (name: string) => Layer;
\end{lstlisting}
レイヤー名を指定することで、レイヤー情報を取得できる。活性化状態はレイヤーの中にisActiveとして格納されている。

\subsubsection{アクティブなレイヤーの取得}
アクティブなレイヤーを取得したい場合はgetActiveLayersというメソッドを提供している。
型は以下のように定義する。
\begin{lstlisting}[caption=getActiveLayersの型]
getActiveLayers: () => Layers;
\end{lstlisting}

\subsubsection{複数のレイヤーのパラメーターを取得する}
複数のレイヤーのパラメーターを取得したい場合はgetLayersParamというメソッドを提供している。
型は以下のように定義する。
\begin{lstlisting}[caption=getLayersParamの型]
getLayersParam: (names: string[]) => any[];
\end{lstlisting}
レイヤー名を配列で指定することで、レイヤーパラムの値を配列で取得できる。

\subsubsection{全てのレイヤーのパラメーターを取得する}
全てのレイヤーのパラメーターを取得したい場合はgetAllLayersParamというメソッドを提供している。
型は以下のように定義する。
\begin{lstlisting}[caption=getAllLayersParamの型]
getAllLayersParams: () => any;
\end{lstlisting}

\subsubsection{アクティブなレイヤーのパラメーターを取得する}
アクティブなレイヤーのパラメーターを取得したい場合はgetActiveLayersParamというメソッドを提供している。
型は以下のように定義する。
\begin{lstlisting}[caption=getActiveLayersParamの型]
getActiveLayersParams: () => any;
\end{lstlisting}

\subsubsection{グループを指定してレイヤーのパラメーターを取得する}
グループを指定してレイヤーのパラメーターを取得したい場合はgetLayerParamsByGroupというメソッドを提供している。
型は以下のように定義する。
\begin{lstlisting}[caption=getLayerParamsByGroupの型]
getLayerParamsByGroup: (group: string) => any;
\end{lstlisting}

\subsection{レイヤーを追加する}
レイヤーを追加するには、addLayerというメソッドを提供している。
型は以下のように定義する。
\begin{lstlisting}[caption=addLayerの型]
addLayer: (layers: Layer[]) => void;
\end{lstlisting}
レイヤーを配列で指定することで、レイヤーを追加できる。

\subsection{レイヤーを削除する}
レイヤーを削除するには、removeLayerというメソッドを提供している。
型は以下のように定義する。
\begin{lstlisting}[caption=removeLayerの型]
removeLayer: (name: string) => void;
\end{lstlisting}
レイヤー名を指定することで、レイヤーを削除できる。

\subsection{レイヤーの活性化を切り替える}
レイヤーの活性化を切り替えるには、toggleLayerというメソッドを提供している。
型は以下のように定義する。
\begin{lstlisting}[caption=toggleLayerの型]
toggleLayer: (name: string) => void;
\end{lstlisting}

\subsection{グループの設定をする}
グループの設定をするには、setGroupConfigというメソッドを提供している。
型は以下のように定義する。
\begin{lstlisting}[caption=setGroupConfigの型]
setGroupConfig: (name: string, config: GroupConfig) => void;
\end{lstlisting}
グループ名とグループの設定を指定することで、グループの設定をすることができる。

\subsection{レイヤーにグループを設定する}
レイヤーにグループを設定するには、setLayerGroupというメソッドを提供している。
型は以下のように定義する。
\begin{lstlisting}[caption=setLayerGroupの型]
setLayerGroup: (name: string | string[], group: string) => void;
\end{lstlisting}
レイヤー名とグループ名を指定することで、レイヤーにグループを設定することができる。

\subsection{レイヤーの活性化条件を設定する}
レイヤーの活性化条件を設定するには、addDependencyというメソッドを提供している。
型は以下のように定義する。
\begin{lstlisting}[caption=addDependencyの型]
addDependency: (name: string, dependencies: DependencyGroup) => void;
\end{lstlisting}
レイヤー名と活性化条件を指定することで、レイヤーの活性化条件を設定することができる。

\expandafter\ifx\csname ifdraft\endcsname\relax
\end{document}
\fi
