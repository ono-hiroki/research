\expandafter\ifx\csname ifdraft\endcsname\relax
\documentclass{jsarticle}
\begin{document}
\fi
\section{評価}
本章では、実装した機能の評価を行う。
実装した機能によって何ができるようになったか、改善されたかを示す。

\subsection{state変数の変化}
ReactCOPでは、レイヤーの活性化情報をuseStateを用いて管理している。state変数はクラスのインスタンスを用いている。
しかし、state変数は参照型のため、state変数の値を変更しても再レンダリングが行われない。
そのため、レイヤーの活性化情報を更新しても、再レンダリングが行われない。
これを解決するために、state変数を参照型のクラスのインスタンスから、イミュータブルなデータに変更する必要がある。
この変更はreact copを大きく刷新することと同等であるため、ReactCOP2ではとして新しく実装することとした。
この実装によりレイヤーに何か操作を加えるたびにnotifyUpdatedLayerState()を呼び出す必要がなくなりコードがより簡潔になった。

\subsection{レイヤーの管理}
ReactCOPでは、レイヤーのパラメーターとlayerの活性化情報を別で管理している。
そのため、レイヤーの活性化とレイヤーのパラメーターで2度レイヤー名を指定する必要がある。
これは、レイヤーの管理が煩雑になる原因となっている。
この問題を解決するために、レイヤーを一つのオブジェクトとして管理することとした。
この実装によりレイヤーの管理が簡単になった。

\subsection{レイヤー操作機能の拡張性}
ReactCOPでは与えられたレイヤーの操作のみを行うことができる。
しかしユーザーによっては、レイヤーの操作を拡張したい場合がある。
ReactCOP2ではレイヤーの操作機能を拡張するために、レイヤーを管理しているuseStateの返り値であるsetLayersとlayersをそのまま提供している。
ユーザーはカスタムフックを自作で実装するよことにより、レイヤーの操作機能を拡張することができる。

\subsection{グループの追加}
ReactCOPでは、レイヤーのグループを設定することができない。
しかし実際にはレイヤーはグループに属していることが多い。
例えば、晴れや雨などの天気のレイヤーは天気のグループに属している。
私達が実際に認識していることを表現するためには、レイヤーのグループを設定することが必要である。
グループを表現することにより私達にとって認識しやすくなる。
ReactCOP2では、レイヤーのグループを設定することができるようになった。

\subsection{グループ内で活性化を排反にする}
ReactCOPでは複数のレイヤーが活性化している状態を許容している。
場合によっては、複数のレイヤーが活性化している状態を許容したくない場合がある。
同時に活性化を許容しているためReactCOPのサンプルアプリである電卓アプリでは以下のように冗長な表現となる。
\begin{lstlisting}[]
// ReactCOP
<Layer condition={layerState.Float && !layerState.Integer}>
\end{lstlisting}
少なくとも私の感覚としては、電卓のモード選択にはFloatかつIntegerが同時に活性化している状態は存在しないため、Floatが活性化していることのみをconditionに設定したい。
ReactCOP2では、グループ内で活性化を排反にすることができるようになった。
\begin{lstlisting}[]
// ReactCOP2
<Layer condition={Float.isActive}>
\end{lstlisting}
このように、グループ内で活性化を排反にすることにより、冗長な表現を簡潔にすることができる。

\subsection{複合層・多層}
ReactCOPでは、複合層・多層を実現することができない。
複合層・多層は、レイヤーの活性化状態を条件として設定することで実現できる。
ReactCOP2では、複合層・多層を実現することができるようになった。

\subsection{テストの追加}
ReactCOPではテストを行っていなかった。
テストは、コードの品質を保つために重要である。
ReactCOP2では、テストを追加した。

\subsection{型の追加}
ReactCOPでは、型を定義していなかった。
型は、コードの品質を保つために重要である。
ReactCOP2では、型を追加した。


\subsubsection{レイヤーパラムの取得}
ReactCOPでは、レイヤーパラムを取得する機能が存在した。
ReactCOPではレイヤーの活性化状態とレイヤーパラムを別で管理している。
必然的にそれらを取得するためのメソッドも別々に用意する必要があった。

ReactCOP2では、レイヤーの活性化状態とレイヤーパラムを一つのオブジェクトとして管理している。
そのため、レイヤーを取得すればレイヤーの活性化状態とレイヤーパラムを取得できる。
代替する機能としてgetLayerParamというメソッドを提供している。
代替機能として用意したものの、レイヤーを取得すれば十分な必要がないかもしれないと考える。
getLayerParamのメリットとしては、レイヤーではなく直接レイヤーのパラムを取得できることである。
しかし、ライブラリを使用する上でたくさんのメソッドがあると、ライブラリの理解が難しくなる。
そのため、レイヤーを取得するメソッドで十分であると考える。

\subsection{レイヤーの追加}
ReactCOPでは明確にレイヤーを追加する機能は存在しなかった。
レイヤーを追加するためには、レイヤーの活性化状態などの設定時にもしレイヤーが存在しない場合は新しくレイヤーを追加するという処理を行っていた。
これは意図しないレイヤーが混入する可能性があるた。
レイヤーを操作することとレイヤーの追加を分けることで、意図しないレイヤーが混入する可能性を減らすことができる。

\subsection{レイヤーの削除}
ReactCOPでは明確にレイヤーを削除する機能は存在しなかった。
必要ないレイヤーが存在すると、レイヤーの管理が煩雑になる。
レイヤーを削除することで、レイヤーの管理が簡単になる。

\subsection{レイヤーの活性化の切り替え}
ReactCOPでは、レイヤーの活性化の切り替えは、もしレイヤーが活性化している場合は非活性化し、非活性化している場合は活性化するという処理を行っていた。
ReactCOP2では、レイヤーの活性化の切り替えを行うためのメソッドを提供している。
コードベースで見ると以下のようになる。
\begin{lstlisting}[]
// ReactCOP
layerState.hoge ? layerManager.deactivateLayer("hoge") : layerManager.activateLayer("hoge");

// ReactCOP2
toggleLayer('hoge')
\end{lstlisting}
ReactCOP2のほうが、コードが簡潔になり読みやすくなっている。

\subsection{レイヤーの活性化}
レイヤーの活性化にはactiveLayerというメソッドを提供している。
レイヤーを活性化するという意味では、activeLayerよりactivateLayerのほうが適切である。
activeLayerでは、アクティブなレイヤーを取得するという意味と間違える可能性がある。
この研究を引き継ぐ際には、activeLayerをactivateLayerに変更することを推奨する。

\subsection{レイヤーの非活性化}
レイヤーの非活性化にはinactiveLayerというメソッドを提供している。
レイヤーを非活性化するという意味では、inactiveLayerよりdeactivateLayerのほうが適切である。
inactiveLayerでは、非アクティブなレイヤーを取得するという意味と間違える可能性がある。
この研究を引き継ぐ際には、inactiveLayerをdeactivateLayerに変更することを推奨する。

\subsection{今後の課題}
本研究ではReactCOPを大きく刷新しReactCOP2を実装した。本章で述べた問題もいくつか残っている。
まだアプリケーションに導入して検証を行っていない。
ReactCOPとの機能の比較により、ReactCOP2の機能が優位であることは議論したが、実際にアプリケーションに導入して検証を行う必要がある。

\expandafter\ifx\csname ifdraft\endcsname\relax
\end{document}
\fi
