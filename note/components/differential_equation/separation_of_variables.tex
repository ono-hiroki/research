\expandafter\ifx\csname ifdraft\endcsname\relax
\documentclass{jsarticle}
\begin{document}
    \fi

% ここに文書の内容を書く


    \section{変数分離形}

    \subsection{基本的な解き方}
    \begin{align}
        y^\prime &= f(x)g(y)\\
        g(y) \neq 0 のとき \\
        \frac{dy}{dx} &= f(x)g(y)\\
        \frac{1}{g(y)}\frac{dy}{dx} &= f(x)dx\\
        \int \frac{1}{g(y)}dy &= \int f(x)dx
    \end{align}

    \subsection{問 $y^\prime = 2xy(y-1)$を解く}
    定数関数$y\equiv0$と$y\equiv1$は解であることが容易にわかる
    $y \neq 0, 1$のとき
    \begin{align}
        y^\prime &= 2xy(y-1)\\
        \frac{dy}{dx} &= 2xy(y-1)\\
        \frac{1}{y(y-1)}dy &= 2xdx\\
        \int \frac{1}{y(y-1)}dy &= \int 2xdx\\
        \int \left(\frac{1}{y-1} - \frac{1}{y}\right)dy &= \int 2xdx\\
        \log|y - 1| - \log|y| &= x^2 + C\\
        \log\left|\frac{y}{y-1}\right| &= x^2 + C\\
        \left| \frac{y}{y-1} \right| &= e^{x^2 + C}\\
        \frac{y}{y-1} &= \pm e^{C}e^{x^2}\\
%        \intertext{あらためてe^{C}をCとおくと (C \neq 0)}
        \intertext{あらためて$e^{C}$をCとおくと $(C \neq 0)$}
        \frac{y}{y-1} &= C e^{x^2}\\
        1 - \frac{1}{y} &= C e^{x^2}\\
        \frac{1}{y} &= 1 - C e^{x^2} \\
        y &= \frac{1}{1 - C e^{x^2}} \\
        \intertext{y= 1が解であることからC = 0のときも解であるから、微分方程式の解はCを任意定数として}
        y &= \frac{1}{1 - C e^{x^2}} または y  \equiv 0
        \intertext{である。}
    \end{align}

    \subsection{問 $y^\prime = e^{2x-3y}$を解く}
    \begin{align}
        y^\prime &= e^{2x-3y}\\
        \frac{dy}{dx} &= e^{2x-3y}\\
        e^{3y}dy &= e^{2x}dx\\
        \int e^{3y}dy &= \int e^{2x}dx\\
        \frac{1}{3}e^{3y} &= \frac{1}{2}e^{2x} + C\\
        e^{3y} &= \frac{3}{2}e^{2x} + C\\
        3y &= \log\left(\frac{3}{2}e^{2x} + C\right)\\
        y &= \frac{1}{3}\log\left(\frac{3}{2}e^{2x} + C\right)
    \end{align}

    \subsection{問 $y^\prime = x^2 y^2$を解く}
    \begin{align}
        \intertext{y $\equiv$ 0は解であることが容易にわかる。y $\neq$ 0のとき}
        y^\prime &= x^2 y^2\\
        \frac{dy}{dx} &= x^2 y^2\\
        \frac{1}{y^2}dy &= x^2 dx\\
        \int \frac{1}{y^2}dy &= \int x^2 dx\\
        -\frac{1}{y} &= \frac{1}{3}x^3 + C\\
        y &= -\frac{3}{x^3} + \frac{3C}{x^3}\\
        y &= \frac{3}{C - x^3}
    \end{align}

    \subsection{$y^\prime = f(ax + by)(b \neq 0)$のときに変数分離形にするやり方}

    \begin{align}
        \intertext{$u(x) = ax + by(x)$とおくと}
        y^\prime &= \frac{u^\prime-a}{b}
        \intertext{であるから、$y^\prime = f(ax + by)$に代入すると}
        \frac{u^\prime-a}{b} &= f(u)\\
        u^\prime &= bf(u) + a \\
        \intertext{となり$x$の関数$u(x)$についての微分方程式になる。}
    \end{align}

    \subsection{問 $y^\prime = \dfrac{x-2y+1}{2x-4y+3}$を解く}


    \expandafter\ifx\csname ifdraft\endcsname\relax
\end{document}
\fi
