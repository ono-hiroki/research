\documentclass[a4paper]{jarticle}
%\usepackage{amsmath}
% サブサブセクションを (1),(2)にする設定
\renewcommand{\thesubsubsection}{(\arabic{subsubsection})}
% (i),(ii)なら \arabic を \roman に変える。    (a),(b)なら \alph

% 大問2の3番目の計算式のラベルを (2.3) にする設定
% 計算式の参照には \eqref{eq:hoge} を使う
\newcommand{\ifdraft}{false}
\makeatletter
\renewcommand{\theequation}{\arabic{subsection}.\arabic{equation}}
\@addtoreset{equation}{subsection}
\makeatother

% --------------------------------------------------------------------
\usepackage{./style/customstyle}


\title{\Huge 微分方程式}%タイトル
\date{\today}%日付
\author{\quad hono}%名前

\begin{document}
    \maketitle
    \pagenumbering{roman}
    \newpage
    \tableofcontents%目次が消したい場合はコメントアウト
    \newpage

    \pagenumbering{arabic}

%%%%%%%%%%%%%%%%%%%%%%%%%%%%%%%%%%%%%%%%%%%%%%%%%%%%%%%%%%%%%%%%%%%%%%%%%%%%%
%%%%%%%%%%%%%%%%%%%%%%%%%%%%%%%%%%%%%%%%%%%%%%%%%%%%%%%%%%%%%%%%%%%%%%%%%%%%%
%%%%%%%%%%%%%%%%%%%%%%%%%%%%%%%%%%%%%%%%%%%%%%%%%%%%%%%%%%%%%%%%%%%%%%%%%%%%%
%以下サンプル%ここから書き始めてください

%セクション(目次に表示されるのは初期設定では)








    \section{変数分離形fdsa}

    \subsection{基本的な解き方}
    \begin{align}
        y^\prime &= f(x)g(y)\\
        g(y) \neq 0 のとき \\
        \frac{dy}{dx} &= f(x)g(y)\\
        \frac{1}{g(y)}\frac{dy}{dx} &= f(x)dx\\
        \int \frac{1}{g(y)}dy &= \int f(x)dx
    \end{align}

    \subsection{問 $y^\prime = 2xy(y-1)$を解く}
    定数関数$y\equiv0$と$y\equiv1$は解であることが容易にわかる
    $y \neq 0, 1$のとき
    \begin{align}
        y^\prime &= 2xy(y-1)\\
        \frac{dy}{dx} &= 2xy(y-1)\\
        \frac{1}{y(y-1)}dy &= 2xdx\\
        \int \frac{1}{y(y-1)}dy &= \int 2xdx\\
        \int \left(\frac{1}{y-1} - \frac{1}{y}\right)dy &= \int 2xdx\\
        \log|y - 1| - \log|y| &= x^2 + C\\
        \log\left|\frac{y}{y-1}\right| &= x^2 + C\\
        \left| \frac{y}{y-1} \right| &= e^{x^2 + C}\\
        \frac{y}{y-1} &= \pm e^{C}e^{x^2}\\
%        \intertext{あらためてe^{C}をCとおくと (C \neq 0)}
        \intertext{あらためて$e^{C}$をCとおくと $(C \neq 0)$}
        \frac{y}{y-1} &= C e^{x^2}\\
        1 - \frac{1}{y} &= C e^{x^2}\\
        \frac{1}{y} &= 1 - C e^{x^2} \\
        y &= \frac{1}{1 - C e^{x^2}} \\
        \intertext{y= 1が解であることからC = 0のときも解であるから、微分方程式の解はCを任意定数として}
        y &= \frac{1}{1 - C e^{x^2}} または y  \equiv 0
        \intertext{である。}
    \end{align}

    \subsection{問 $y^\prime = e^{2x-3y}$を解く}
    \begin{align}
        y^\prime &= e^{2x-3y}\\
        \frac{dy}{dx} &= e^{2x-3y}\\
        e^{3y}dy &= e^{2x}dx\\
        \int e^{3y}dy &= \int e^{2x}dx\\
        \frac{1}{3}e^{3y} &= \frac{1}{2}e^{2x} + C\\
        e^{3y} &= \frac{3}{2}e^{2x} + C\\
        3y &= \log\left(\frac{3}{2}e^{2x} + C\right)\\
        y &= \frac{1}{3}\log\left(\frac{3}{2}e^{2x} + C\right)
    \end{align}

    \subsection{問 $y^\prime = x^2 y^2$を解く}
    \begin{align}
        \intertext{y $\equiv$ 0は解であることが容易にわかる。y $\neq$ 0のとき}
        y^\prime &= x^2 y^2\\
        \frac{dy}{dx} &= x^2 y^2\\
        \frac{1}{y^2}dy &= x^2 dx\\
        \int \frac{1}{y^2}dy &= \int x^2 dx\\
        -\frac{1}{y} &= \frac{1}{3}x^3 + C\\
        y &= -\frac{3}{x^3} + \frac{3C}{x^3}\\
        y &= \frac{3}{C - x^3}
    \end{align}

    \subsection{$y^\prime = f(ax + by)(b \neq 0)$のときに変数分離形にするやり方}

    \begin{align}
        \intertext{$u(x) = ax + by(x)$とおくと}
        y^\prime &= \frac{u^\prime-a}{b}
        \intertext{であるから、$y^\prime = f(ax + by)$に代入すると}
        \frac{u^\prime-a}{b} &= f(u)\\
        u^\prime &= bf(u) + a \\
        \intertext{となり$x$の関数$u(x)$についての微分方程式になる。}
    \end{align}

    \subsection{問 $y^\prime = \dfrac{x-2y+1}{2x-4y+3}$を解く}
    \expandafter\ifx\csname ifdraft\endcsname\relax
\documentclass{jsarticle}
\begin{document}
    \fi

% ここに文書の内容を書く


%囲い
    \begin{itembox}[l]{囲いのタイトル左版}
        複数行の四角の囲い.
    \end{itembox}

    \begin{itembox}[c]{囲いのタイトル中央版}
        複数行の四角の囲い. 囲いのタイトル左版
    \end{itembox}

    \begin{itembox}[r]{囲いのタイトル右版}
        複数行の四角の囲い.
    \end{itembox}

%改ページ
    \newpage

%箇条書き
    \begin{itemize}
        \item 箇条書き
        \item 箇条書き
        \item 箇条書き
    \end{itemize}

    \begin{itemize}
        \item 箇条書き
        \item 箇条書き
        \item 箇条書き

        \begin{itemize}
            \item 箇条書きの中にも箇条書きを入れられる
            \item 箇条書きの中にも箇条書きを入れられる
            \item 箇条書きの中にも箇条書きを入れられる

            \begin{itemize}
                \item 箇条書きの中にも箇条書きを入れられる
                \item 箇条書きの中にも箇条書きを入れられる
                \item 箇条書きの中にも箇条書きを入れられる

                \begin{itemize}
                    \item 箇条書きの中にも箇条書きを入れられる
                    \item 箇条書きの中にも箇条書きを入れられる
                    \item 箇条書きの中にも箇条書きを入れられる
                \end{itemize}

            \end{itemize}

        \end{itemize}

    \end{itemize}

%番号付き箇条書き
    \begin{enumerate}
        \item 数字つき箇条書き
        \item 数字つき箇条書き
        \item 数字つき箇条書き
    \end{enumerate}

    \begin{enumerate}
        \item 数字つき箇条書き
        \item 数字つき箇条書き
        \item 数字つき箇条書き

        \begin{enumerate}
            \item 数字つき箇条書き
            \item 数字つき箇条書き
            \item 数字つき箇条書き

            \begin{enumerate}
                \item 数字つき箇条書き
                \item 数字つき箇条書き
                \item 数字つき箇条書き

                \begin{enumerate}
                    \item 数字つき箇条書き
                    \item 数字つき箇条書き
                    \item 数字つき箇条書き
                \end{enumerate}

            \end{enumerate}

        \end{enumerate}

    \end{enumerate}

%見出し付き箇条書き
    \begin{description}
        \item[あいうえお] 見出し付き箇条書き
        \item[かきくけこ] 見出し付き箇条書き
        \item[さしすせそ] 見出し付き箇条書き
    \end{description}


%ラベルは\ref{Label}のように書くと数字だけ表示される(2回コンパイルが必要). {括弧}の中身は任意(英語推奨).


%参考資料記載
    \begin{thebibliography}{1}
%著者, タイトル, 出版社, 現在の版の第1刷が発行された年. など, 詳しくはググる
        \bibitem{kaggle} 門脇大輔 他, Kaggleで勝つデータ分析の技術 : 2019.10.22
%\bibitem{ラベル名}, \cite{ラベル名}で呼び出せる(2回コンパイルが必要)
    \end{thebibliography}

    \subsection{サブセクションは大問に対応する}

    \subsubsection{}

    ここは問1の(1)。
    サブサブセクションは小問に対応。

    \subsubsection{}

    ここは問1の(2)。

% --------------------------------------------------------------------
% 問2

    \subsection{計算式のラベルと参照について}

    \subsubsection{}

    ここは問2の(1)。
    簡単な計算
    \begin{align}
        a &= b \notag \\
        c &= d \label{eq:hoge} %これは問2の最初の計算式なので (2.1) となる
    \end{align}

    \subsubsection{}

    ここは問2の(2)。
    計算式を参照する。\eqref{eq:hoge}は(2.1)みたいになる。

    \expandafter\ifx\csname ifdraft\endcsname\relax
\end{document}
\fi
%    \expandafter\ifx\csname ifdraft\endcsname\relax
\documentclass{jsarticle}
\begin{document}
    \fi

% ここに文書の内容を書く

    \section{例題集}

    \subsection{問 次の関数が$y^{\prime\prime} - 2y^\prime - 3y = 0$の微分方程式の解であることを代入により確かめよ}

%    \begin{enumerate}
%        \item[\rm(\hspace{.18em}i\hspace{.18em})] $y^{\prime\prime} - 2y^\prime - 3y = 0$
%    \end{enumerate}

    \subsubsection{$e^{-x}$}

    \subsubsection{$e^{-3x}$}

    \subsubsection{$c_1 e^{-x} + c_2 e^{-3x}$}

    \subsection{解答}

    \subsubsection{$e^{-x}$}
    \begin{align}
        e^{-x} + 2e^{-x} - 3e^{-x} &= 0\\
    \end{align}

    \subsubsection{$e^{-3x}$}
    \begin{align}
        9e^{-3x} - 6e^{-3x} - 3e^{-3x} &= 0\\
    \end{align}

    \subsubsection{$c_1 e^{-x} + c_2 e^{-3x}$}
    \begin{align}
        c_1(e^{-x} + 2e^{-x} - 3e^{-x}) + c_2(9e^{-3x} - 6e^{-3x} - 3e^{-3x}) \\
        = 0
    \end{align}

    \expandafter\ifx\csname ifdraft\endcsname\relax
\end{document}
\fi

\end{document}
