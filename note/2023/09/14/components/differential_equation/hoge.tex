\expandafter\ifx\csname ifdraft\endcsname\relax
\documentclass{jsarticle}
\begin{document}
    \fi

% ここに文書の内容を書く


%囲い
    \begin{itembox}[l]{囲いのタイトル左版}
        複数行の四角の囲い.
    \end{itembox}

    \begin{itembox}[c]{囲いのタイトル中央版}
        複数行の四角の囲い. 囲いのタイトル左版
    \end{itembox}

    \begin{itembox}[r]{囲いのタイトル右版}
        複数行の四角の囲い.
    \end{itembox}

%改ページ
    \newpage

%箇条書き
    \begin{itemize}
        \item 箇条書き
        \item 箇条書き
        \item 箇条書き
    \end{itemize}

    \begin{itemize}
        \item 箇条書き
        \item 箇条書き
        \item 箇条書き

        \begin{itemize}
            \item 箇条書きの中にも箇条書きを入れられる
            \item 箇条書きの中にも箇条書きを入れられる
            \item 箇条書きの中にも箇条書きを入れられる

            \begin{itemize}
                \item 箇条書きの中にも箇条書きを入れられる
                \item 箇条書きの中にも箇条書きを入れられる
                \item 箇条書きの中にも箇条書きを入れられる

                \begin{itemize}
                    \item 箇条書きの中にも箇条書きを入れられる
                    \item 箇条書きの中にも箇条書きを入れられる
                    \item 箇条書きの中にも箇条書きを入れられる
                \end{itemize}

            \end{itemize}

        \end{itemize}

    \end{itemize}

%番号付き箇条書き
    \begin{enumerate}
        \item 数字つき箇条書き
        \item 数字つき箇条書き
        \item 数字つき箇条書き
    \end{enumerate}

    \begin{enumerate}
        \item 数字つき箇条書き
        \item 数字つき箇条書き
        \item 数字つき箇条書き

        \begin{enumerate}
            \item 数字つき箇条書き
            \item 数字つき箇条書き
            \item 数字つき箇条書き

            \begin{enumerate}
                \item 数字つき箇条書き
                \item 数字つき箇条書き
                \item 数字つき箇条書き

                \begin{enumerate}
                    \item 数字つき箇条書き
                    \item 数字つき箇条書き
                    \item 数字つき箇条書き
                \end{enumerate}

            \end{enumerate}

        \end{enumerate}

    \end{enumerate}

%見出し付き箇条書き
    \begin{description}
        \item[あいうえお] 見出し付き箇条書き
        \item[かきくけこ] 見出し付き箇条書き
        \item[さしすせそ] 見出し付き箇条書き
    \end{description}


%ラベルは\ref{Label}のように書くと数字だけ表示される(2回コンパイルが必要). {括弧}の中身は任意(英語推奨).


%参考資料記載
    \begin{thebibliography}{1}
%著者, タイトル, 出版社, 現在の版の第1刷が発行された年. など, 詳しくはググる
        \bibitem{kaggle} 門脇大輔 他, Kaggleで勝つデータ分析の技術 : 2019.10.22
%\bibitem{ラベル名}, \cite{ラベル名}で呼び出せる(2回コンパイルが必要)
    \end{thebibliography}

    \subsection{サブセクションは大問に対応する}

    \subsubsection{}

    ここは問1の(1)。
    サブサブセクションは小問に対応。

    \subsubsection{}

    ここは問1の(2)。

% --------------------------------------------------------------------
% 問2

    \subsection{計算式のラベルと参照について}

    \subsubsection{}

    ここは問2の(1)。
    簡単な計算
    \begin{align}
        a &= b \notag \\
        c &= d \label{eq:hoge} %これは問2の最初の計算式なので (2.1) となる
    \end{align}

    \subsubsection{}

    ここは問2の(2)。
    計算式を参照する。\eqref{eq:hoge}は(2.1)みたいになる。

    \expandafter\ifx\csname ifdraft\endcsname\relax
\end{document}
\fi