\documentclass[a4paper]{jarticle}
\usepackage{amsmath}
\newcommand{\ifdraft}{false}
\makeatletter
\renewcommand{\theequation}{\arabic{subsection}.\arabic{equation}}
\@addtoreset{equation}{subsection}
\makeatother

% --------------------------------------------------------------------
\usepackage{./style/customstyle}


\title{\Huge 微分積分学}%タイトル
\date{\today}%日付
\author{\quad hono}%名前



\begin{document}
    \maketitle
    \pagenumbering{roman}
    \newpage
    \tableofcontents
    \newpage
    \pagenumbering{arabic}


%    \expandafter\ifx\csname ifdraft\endcsname\relax
\documentclass{jsarticle}
\begin{document}
    \fi

% ここに文書の内容を書く


    \section{変数分離形}

    \subsection{基本的な解き方}
    \begin{align}
        y^\prime &= f(x)g(y)\\
        g(y) \neq 0 のとき \\
        \frac{dy}{dx} &= f(x)g(y)\\
        \frac{1}{g(y)}\frac{dy}{dx} &= f(x)dx\\
        \int \frac{1}{g(y)}dy &= \int f(x)dx
    \end{align}

    \subsection{問 $y^\prime = 2xy(y-1)$を解く}
    定数関数$y\equiv0$と$y\equiv1$は解であることが容易にわかる
    $y \neq 0, 1$のとき
    \begin{align}
        y^\prime &= 2xy(y-1)\\
        \frac{dy}{dx} &= 2xy(y-1)\\
        \frac{1}{y(y-1)}dy &= 2xdx\\
        \int \frac{1}{y(y-1)}dy &= \int 2xdx\\
        \int \left(\frac{1}{y-1} - \frac{1}{y}\right)dy &= \int 2xdx\\
        \log|y - 1| - \log|y| &= x^2 + C\\
        \log\left|\frac{y}{y-1}\right| &= x^2 + C\\
        \left| \frac{y}{y-1} \right| &= e^{x^2 + C}\\
        \frac{y}{y-1} &= \pm e^{C}e^{x^2}\\
%        \intertext{あらためてe^{C}をCとおくと (C \neq 0)}
        \intertext{あらためて$e^{C}$をCとおくと $(C \neq 0)$}
        \frac{y}{y-1} &= C e^{x^2}\\
        1 - \frac{1}{y} &= C e^{x^2}\\
        \frac{1}{y} &= 1 - C e^{x^2} \\
        y &= \frac{1}{1 - C e^{x^2}} \\
        \intertext{y= 1が解であることからC = 0のときも解であるから、微分方程式の解はCを任意定数として}
        y &= \frac{1}{1 - C e^{x^2}} または y  \equiv 0
        \intertext{である。}
    \end{align}

    \subsection{問 $y^\prime = e^{2x-3y}$を解く}
    \begin{align}
        y^\prime &= e^{2x-3y}\\
        \frac{dy}{dx} &= e^{2x-3y}\\
        e^{3y}dy &= e^{2x}dx\\
        \int e^{3y}dy &= \int e^{2x}dx\\
        \frac{1}{3}e^{3y} &= \frac{1}{2}e^{2x} + C\\
        e^{3y} &= \frac{3}{2}e^{2x} + C\\
        3y &= \log\left(\frac{3}{2}e^{2x} + C\right)\\
        y &= \frac{1}{3}\log\left(\frac{3}{2}e^{2x} + C\right)
    \end{align}

    \subsection{問 $y^\prime = x^2 y^2$を解く}
    \begin{align}
        \intertext{y $\equiv$ 0は解であることが容易にわかる。y $\neq$ 0のとき}
        y^\prime &= x^2 y^2\\
        \frac{dy}{dx} &= x^2 y^2\\
        \frac{1}{y^2}dy &= x^2 dx\\
        \int \frac{1}{y^2}dy &= \int x^2 dx\\
        -\frac{1}{y} &= \frac{1}{3}x^3 + C\\
        y &= -\frac{3}{x^3} + \frac{3C}{x^3}\\
        y &= \frac{3}{C - x^3}
    \end{align}

    \subsection{$y^\prime = f(ax + by)(b \neq 0)$のときに変数分離形にするやり方}

    \begin{align}
        \intertext{$u(x) = ax + by(x)$とおくと}
        y^\prime &= \frac{u^\prime-a}{b}
        \intertext{であるから、$y^\prime = f(ax + by)$に代入すると}
        \frac{u^\prime-a}{b} &= f(u)\\
        u^\prime &= bf(u) + a \\
        \intertext{となり$x$の関数$u(x)$についての微分方程式になる。}
    \end{align}

    \subsection{問 $y^\prime = \dfrac{x-2y+1}{2x-4y+3}$を解く}


    \expandafter\ifx\csname ifdraft\endcsname\relax
\end{document}
\fi

    \begin{itembox}[l]{定義 関数の極限}
        関数$y=f(x)$において、$x$が$a$と異なる値をとりながら限りなく$a$に近づくとき、
        関数$f(x)$の値がbに限りなく近づくならば「$x$が$a$に近づくときの$f(x)$の極限値は$b$である」といい
        \begin{equation}
            \lim_{x \to a} f(x) = b
        \end{equation}
        あるいは、
        \begin{equation}
            f(x) \to b \quad (x \to a)
        \end{equation}
        と書く
    \end{itembox}
    ※極限の表現が曖昧なので論理的ではない

    \begin{itembox}[l]{定義 左側極限値(右側極限値)}
        $x \to a-0 \quad (x \to a + 0)$のとき$f(x)$の値が$\alpha \quad \beta$に限りなく近づくとき、
        \begin{equation}
            \lim_{x \to a-0} f(x) = \alpha \quad (\lim_{x \to a+0} f(x) = \beta)
        \end{equation}
        と表し、$\alpha \quad (\beta)$を$f(x)$の$a$における左極限値(右極限値)という
    \end{itembox}

    \begin{itembox}[l]{定理 極限の性質}
        $\alpha, \beta$を定数とするとき、$\lim_{x \to a} f(x) = \alpha, \lim_{x \to a} g(x) = \beta$であるとき、次が成り立つ
        \begin{enumerate}
            \item $\lim_{x \to a} k \cdot f(x) = k \cdot \alpha \quad (kは定数)$
            \item $\lim_{x \to a} (f(x) \pm g(x)) = \alpha \pm \beta$
            \item $\lim_{x \to a} (f(x) \cdot g(x)) = \alpha \cdot \beta$
            \item $\lim_{x \to a} \frac{f(x)}{g(x)} = \frac{\alpha}{\beta} \quad (\beta \neq 0)$
        \end{enumerate}
    \end{itembox}
%    TODO 証明



    \begin{itembox}[l]{定理 極限の性質2}
        $\lim_{x \to a} f(x) = \alpha, \lim_{x \to a} g(x) = \beta$であるとき、次が成り立つ
        \begin{enumerate}
            \item $x=a$の近くで$f(x) \leq g(x)$であるとき、$\alpha \leq \beta$
            \item $x=a$の近くで$f(x) \geq h(x) \geq g(x)$ かつ $\alpha = \beta$のとき、$\lim_{x \to a} h(x) = \alpha = \beta$
        \end{enumerate}
    \end{itembox}
%    TODO 証明



    \begin{itembox}[l]{定義 連続}
        $x=a$において$f(x)$が定義されているとき、
        \begin{equation}
            \lim_{x \to a} f(x) = f(a)
        \end{equation}
        が成り立つとき、$f(x)$は$x=a$において連続であるという
    \end{itembox}

    \begin{itembox}[l]{定義 右連続(左連続)}
        $x=a$において$f(x)$が定義されているとき、
        \begin{equation}
            \lim_{x \to a+0} f(x) = f(a) \quad (\lim_{x \to a-0} f(x) = f(a))
        \end{equation}
        が成り立つとき、$f(x)$は$x=a$において右連続(左連続)であるという

    \end{itembox}

    \begin{itembox}[l]{定理}
        $k$を定数とする。$f(x), g(x)$はともに$x=a$で連続ならば
        \begin{enumerate}
            \item $k \cdot f(x)$も$x=a$で連続
            \item $f(x) \pm g(x)$も$x=a$で連続
            \item $f(x) \cdot g(x)$も$x=a$で連続
        \end{enumerate}
        も連続である。また$g(x) \neq 0$のとき、$f(x) / g(x)$も$x=a$で連続である
    \end{itembox}

    \begin{itembox}[l]{定理}
        \begin{enumerate}
            \item $\lim_{x \to 0} \dfrac{\sin x}{x} = 1$
            \item $\lim_{x \to \pm \infty} (1 + \frac{1}{x})^x = e$
            \item $\lim_{x \to 0} \dfrac{\log (1 + x)}{x} = 1$
            \item $\lim_{x \to 0} \dfrac{e^x - 1}{x} = 1$
        \end{enumerate}
    \end{itembox}

    \begin{itembox}[l]{定理 合成関数の連続性}
        $f(x)$が$x=a$で連続、$b=f(a)$とする。
        また$y=g(x)$が$x=b$で連続であるとき、$y=g \circ f(x)$は$x=a$で連続である
    \end{itembox}

    \begin{itembox}[l]{定理 逆関数の連続性}
        $f(x)$が閉区間$[a, b]$上で狭義単調増加ならば、
        逆関数$y=f^{-1}(x)$は閉区間$[f(a), f(b)]$上で連続で狭義単調増加である。
        狭義単調減少の場合も同様
    \end{itembox}

    \begin{itembox}[l]{定理 有界閉区間上の連続関数の最大値と最小値}
        閉区間$[a, b]$上で連続な関数$f(x)$は、$[a, b]$上で最大値と最小値をもつ
    \end{itembox}

    \begin{itembox}[l]{定理 中間値の定理}
        $f(x)$が閉区間$[a, b]$上で連続、$f(a) \neq f(b)$であるとき、
        $f(a)$と$f(b)$の間の任意の値$c$に対して、
        \[f(c) = \mu \quad (a < c < b)\]
        を満たす点が少なくとも1つ存在する
    \end{itembox}

    \begin{itembox}[l]{定義 微分可能}
        $\lim_{h \to 0} \dfrac{f(x+h) - f(x)}{h} = \lim_{x \to a} \dfrac{f(x) - f(a)}{x - a}$が存在するとき、
        $f(x)$は$x=a$で微分可能であるという。この極限値を$f^{\prime}(a)$と書き、$f(x)$の$x=a$における微分係数という
        $f(x)$が区間$I$上で微分可能であるとき、$f(x)$は$I$上で微分可能であるという
    \end{itembox}


    \begin{itembox}[l]{定義 右微分可能(左微分可能)}
        $\lim_{h \to a-0} \dfrac{f(x+h) - f(x)}{h}, \lim_{x \to a+0} \dfrac{f(x) - f(a)}{x - a}$が存在するとき、
        それぞれ$f(x)$は$x=a$において右微分可能(左微分可能)であるという。
        これらの極限値を$f^{\prime}_{+}(a), f^{\prime}_{-}(a)$と書き、$x=a$における右微分係数(左微分係数)という
    \end{itembox}

    \begin{itembox}[l]{定理 微分可能ならば連続}
        関数$f(x)$が$x=a$で微分可能ならば、$f(x)$は$x=a$で連続である
    \end{itembox}

    \begin{itembox}[l]{定理 合成関数の微分法}
        $f(x), g(x)$は微分可能とする。このとき次が成り立つ。
        \begin{enumerate}
            \item $(k \cdot f(x))^{\prime} = k \cdot f^{\prime}(x) \quad (kは定数)$
            \item $(f(x) \pm g(x))^{\prime} = f^{\prime}(x) \pm g^{\prime}(x)$
            \item $(f(x) \cdot g(x))^{\prime} = f^{\prime}(x) \cdot g(x) + f(x) \cdot g^{\prime}(x)$
            \item $\left(\dfrac{f(x)}{g(x)}\right)^{\prime} = \dfrac{f^{\prime}(x) \cdot g(x) - f(x) \cdot g^{\prime}(x)}{g(x)^2} \quad (g(x) \neq 0)$
        \end{enumerate}
    \end{itembox}

    \begin{itembox}[l]{定理 合成関数の微分公式}
        $y=f(x)$が$x$について微分の安濃であり、$z=g(y)$について微分可能ならば、
        合成関数$z=g(f(x))$は$x$について微分可能であり、次の式が成り立つ。
        \begin{equation}
            \frac{dz}{dx} = \frac{dz}{dy} \cdot \frac{dy}{dx} = g^{\prime}(f(x)) \cdot f^{\prime}(x)
        \end{equation}
    \end{itembox}

    \begin{itembox}[l]{定理 逆関数の微分公式}
        $x=f(y)$が微分可能な単調関数ならば、その逆関数$y=f^{-1}(x)$は$f^{\prime}(y) \neq 0$を満たす点$x$で
        微分可能であり、次の式が成り立つ。
        \begin{equation}
            \frac{dy}{dx} = \frac{1}{\frac{dx}{dy}}
        \end{equation}
    \end{itembox}

    \begin{itembox}[l]{定理 バラメータ表示された関数の微分公式}
        $x=\phi(t), y=\psi(t)$が$t$について微分可能な関数で、$\psi(t)$の逆関数が定義され
        $\phi^{\prime}(t) \neq 0$であるとき、$y$は$x$の関数として微分可能であり、次の式が成り立つ。
        \begin{equation}
            \frac{dy}{dx} = \frac{\frac{dy}{dt}}{\frac{dx}{dt}} = \frac{\psi^{\prime}(t)}{\phi^{\prime}(t)} \quad (\phi^{\prime}(t) \neq 0)
        \end{equation}
    \end{itembox}

    \begin{itembox}[l]{定義 高階導関数}
       $y=f(x)$の導関数$f^{\prime}(x)$がさらに微分可能であるとき、
        $f(x)$は2回微分可能であるといい、$f^{\prime \prime}(x)$を$f(x)$の2階導関数という。
        一般に、$f(x)$が$n$回微分可能であるとき、$f(x)$をn回微分することによって得られる関数を
        $f(x)$のn階導関数といい、$f^{(n)}(x)$と表す。
    \end{itembox}

%    \expandafter\ifx\csname ifdraft\endcsname\relax
\documentclass{jsarticle}
\begin{document}
    \fi

% ここに文書の内容を書く


%囲い
    \begin{itembox}[l]{囲いのタイトル左版}
        複数行の四角の囲い.
    \end{itembox}

    \begin{itembox}[c]{囲いのタイトル中央版}
        複数行の四角の囲い. 囲いのタイトル左版
    \end{itembox}

    \begin{itembox}[r]{囲いのタイトル右版}
        複数行の四角の囲い.
    \end{itembox}

%改ページ
    \newpage

%箇条書き
    \begin{itemize}
        \item 箇条書き
        \item 箇条書き
        \item 箇条書き
    \end{itemize}

    \begin{itemize}
        \item 箇条書き
        \item 箇条書き
        \item 箇条書き

        \begin{itemize}
            \item 箇条書きの中にも箇条書きを入れられる
            \item 箇条書きの中にも箇条書きを入れられる
            \item 箇条書きの中にも箇条書きを入れられる

            \begin{itemize}
                \item 箇条書きの中にも箇条書きを入れられる
                \item 箇条書きの中にも箇条書きを入れられる
                \item 箇条書きの中にも箇条書きを入れられる

                \begin{itemize}
                    \item 箇条書きの中にも箇条書きを入れられる
                    \item 箇条書きの中にも箇条書きを入れられる
                    \item 箇条書きの中にも箇条書きを入れられる
                \end{itemize}

            \end{itemize}

        \end{itemize}

    \end{itemize}

%番号付き箇条書き
    \begin{enumerate}
        \item 数字つき箇条書き
        \item 数字つき箇条書き
        \item 数字つき箇条書き
    \end{enumerate}

    \begin{enumerate}
        \item 数字つき箇条書き
        \item 数字つき箇条書き
        \item 数字つき箇条書き

        \begin{enumerate}
            \item 数字つき箇条書き
            \item 数字つき箇条書き
            \item 数字つき箇条書き

            \begin{enumerate}
                \item 数字つき箇条書き
                \item 数字つき箇条書き
                \item 数字つき箇条書き

                \begin{enumerate}
                    \item 数字つき箇条書き
                    \item 数字つき箇条書き
                    \item 数字つき箇条書き
                \end{enumerate}

            \end{enumerate}

        \end{enumerate}

    \end{enumerate}

%見出し付き箇条書き
    \begin{description}
        \item[あいうえお] 見出し付き箇条書き
        \item[かきくけこ] 見出し付き箇条書き
        \item[さしすせそ] 見出し付き箇条書き
    \end{description}


%ラベルは\ref{Label}のように書くと数字だけ表示される(2回コンパイルが必要). {括弧}の中身は任意(英語推奨).


%参考資料記載
    \begin{thebibliography}{1}
%著者, タイトル, 出版社, 現在の版の第1刷が発行された年. など, 詳しくはググる
        \bibitem{kaggle} 門脇大輔 他, Kaggleで勝つデータ分析の技術 : 2019.10.22
%\bibitem{ラベル名}, \cite{ラベル名}で呼び出せる(2回コンパイルが必要)
    \end{thebibliography}

    \subsection{サブセクションは大問に対応する}

    \subsubsection{}

    ここは問1の(1)。
    サブサブセクションは小問に対応。

    \subsubsection{}

    ここは問1の(2)。

% --------------------------------------------------------------------
% 問2

    \subsection{計算式のラベルと参照について}

    \subsubsection{}

    ここは問2の(1)。
    簡単な計算
    \begin{align}
        a &= b \notag \\
        c &= d \label{eq:hoge} %これは問2の最初の計算式なので (2.1) となる
    \end{align}

    \subsubsection{}

    ここは問2の(2)。
    計算式を参照する。\eqref{eq:hoge}は(2.1)みたいになる。

    \expandafter\ifx\csname ifdraft\endcsname\relax
\end{document}
\fi
%    \expandafter\ifx\csname ifdraft\endcsname\relax
\documentclass{jsarticle}
\begin{document}
    \fi

% ここに文書の内容を書く

    \section{例題集}

    \subsection{問 次の関数が$y^{\prime\prime} - 2y^\prime - 3y = 0$の微分方程式の解であることを代入により確かめよ}

%    \begin{enumerate}
%        \item[\rm(\hspace{.18em}i\hspace{.18em})] $y^{\prime\prime} - 2y^\prime - 3y = 0$
%    \end{enumerate}

    \subsubsection{$e^{-x}$}

    \subsubsection{$e^{-3x}$}

    \subsubsection{$c_1 e^{-x} + c_2 e^{-3x}$}

    \subsection{解答}

    \subsubsection{$e^{-x}$}
    \begin{align}
        e^{-x} + 2e^{-x} - 3e^{-x} &= 0\\
    \end{align}

    \subsubsection{$e^{-3x}$}
    \begin{align}
        9e^{-3x} - 6e^{-3x} - 3e^{-3x} &= 0\\
    \end{align}

    \subsubsection{$c_1 e^{-x} + c_2 e^{-3x}$}
    \begin{align}
        c_1(e^{-x} + 2e^{-x} - 3e^{-x}) + c_2(9e^{-3x} - 6e^{-3x} - 3e^{-3x}) \\
        = 0
    \end{align}

    \expandafter\ifx\csname ifdraft\endcsname\relax
\end{document}
\fi

\end{document}
